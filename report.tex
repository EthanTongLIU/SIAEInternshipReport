\documentclass{SIAEInternshipReport}


% ------------用户自定义命令方放置于此--------------------------


\begin{document}

% -------------------填写基本信息-------------------------------

\internshiptype{\bluecollar} % --> 实习类别,需要填写。 蓝领实习填写 \bluecollar ,技术实习填写 \technical ,专业实习填写 \pfe

\internshiplocation{XXXXXXXXX公司} % --> 实习地点,填写

\internshipstarttime{2019.07.01} % --> 实习开始时间,填写

\internshipendtime{2019.08.23} % --> 实习结束时间,填写

\author{XXX} % --> 姓名,填写

\studentid{XXXXXXXXXX} % --> 学号,填写

\currentyear{2019} % --> 报告撰写时间,年份,填写

\currentmonth{9} % --> 报告撰写时间,月份,填写

\title{ 题目(中文)} % --> 报告中文题目,填写

\englishtitle{ TITLE(ENGLISH) } % --> 报告英文题目,填写

\abstract{ % --> 中文摘要,填写
    摘要用于对实习过程进行大致描述,包括实习单位、实习内容、个人收获等,对以技术性较强的部分,视其在实习中所占比例可给出进一步介绍。

    摘要内容要求条理清晰、叙述简练,避免综述或介绍性质内容,着重围绕个人的实习过程进行介绍。使读者能够通过摘要了解实习大致过程。

    摘要中所涉及的单位均为国际或行业标准单位,一般不应出现公式和图表。英文缩写、代号、简称等首次出现时应给出全称。

    对于进行技术实习的同学,应突出描述实习过程中所从事的技术活动,包括:该项技术或方法的由来、作用、方案设计、实施过程、效果评估等。

    字数:300-500
}

\keywords{ % --> 中文关键词,填写
   3-5个
}

\englishabstract{ % --> 英文摘要,填写
    英文摘要另起一页,与中文摘要对应。
}

\englishkeywords{ % --> 英文关键词,填写
    (与上文之间空一行,内容应与中文“关键词”一致。词间用分号间隔,尾不加标点。)
}





%-------------------生成标题页和目录页---------------------------------------------

\makecover[]

\makeheaderandfooter[]

\frontmatter


\begin{spacing}{1.0}
\tableofcontents
\thispagestyle{empty}
\end{spacing}

\mainmatter

% \fancypagestyle{plain}{

%      \fancyhf{}
%      \lhead{}
%      \chead{ \zihao{5} 中国民航大学中欧航空工程师学院2019年技术实习报告 }
%      \cfoot{\thepage}
%      \renewcommand{\headrule}{\hrule width\headwidth}
%      \renewcommand{\headrulewidth}{0.4pt}

% }

\setlength{\baselineskip}{20pt}



% --------------------导入正文,根据个人需要自行调整-----------------------------------

% !TeX root = ../report.tex

\chapter{模板使用说明}

\section{模板简介}

本模板旨在为中欧学院同学提供一套标准规范的蓝领、技术、专业实习报告写作模板,减少同学们浪费在调整文献格式上的时间,故匆忙中制作了这套模板,模板按照《【教学(2017)8号】工程师阶段蓝领实习、技术实习答辩安排及要求通知》附件1所述要求制作,符合其所述规范,同学们可使用本模板生成实习报告,可以摆脱word排版的诸多弊端,譬如不稳定,格式调整繁琐等问题,同时提供了更加简便的特殊文字符号插入能力和多种定制环境支持。

\section{编译方法}

由于本模板中文处理采用 \verb xeCJK 宏包 ,所以本模板\textbf{\color{red}{必须使用 XeLaTeX 引擎编译}}。本模板是在 \verb texlive 下开发制作的,是目前最优的 \LaTeX 内核,故如果本模板在个人计算机上编译不通过,{\color{red}{\textbf{推荐安装} \verb texlive }}。本模板源文件采用 UTF8 编码,如果出现乱码,那是由于你采用了 GBK 编码的编译器,\textbf{\color{red}{希望你卸载 CTEX 套装,卸载 WinEdt 编辑器,使用我推荐的内核与编辑器}}。

{\bfseries 如何采用 XeLaTeX 引擎编译?}
\begin{itemize}
    \item 方法一:在将本机 \verb texlive 添加到环境变量的前提下,双击本模板源文件根目录下的  \verb compile&clean.bat 文件实现编译,如果编译不通过,说明你所输入的源码存在语法或其他错误,请认真检查后调试至通过。

    \item 方法二:使用本机自带 \LaTeX 编辑器,如 \verb SublimeText ,\verb Texmaker 等,自行切换至 XeLaTeX 编译引擎即可。
\end{itemize}


\section{源文件结构}

源文件根目录下包含三个文件夹及两个文档,对各个文件及文件夹的解释如下:

\begin{itemize}
    \item {\bfseries body}

    存放文章的各个章节,在编译时将本部分的文件自动调入主文档。每一章节分开编写,在本部分中填写。文件夹中还包括参考文献、附录、致谢、法语总结和附件。

    \item {\bfseries figure}

    图片路径,所有图片存入本文件夹中,在使用时自动调入。

    \item {\bfseries fonts}

    存放应用的字体。

    \item {\bfseries report.tex}

    主文件,在编译时应该编译此文件。

    \item {\bfseries SIAEInternshipReport.cls}

    定制的实习报告文档类,包含文档基本设置以及封面设计等操作。

\end{itemize}

本模板源文件拓扑结构如图所示:

\begin{Verbatim}[frame=single, framesep=5mm, samepage=false, baselinestretch=1.0]
root --|body --|chap1.tex
       |       |chap2.tex
       |       |chap3.tex
       |       |chap4.tex
       |       |chap5.tex
       |       |reference.tex (参考文献)
       |       |acknowlegement.tex (致谢)
       |       |appendix.tex (附录)
       |       |resume.tex (法文摘要)
       |       |attachement.tex (附件)
       |
       |figure (图片路径)
       |
       |fonts (字体)
       |
       |report.tex (主文件)
       |
       |SIAEInternshipReport.cls (定制文档类)
\end{Verbatim}


% !TeX root = ../report.tex

\chapter{常见环境}

以下所有示例的源码可以在对应的 \TeX 源文件中找到。

\section{字号}

    \begin{table}[H]
    \centering
    \topcaption{中文字号设置}
    \label{tab:zihao-zh}
    \begin{tabular}{lll}
    \toprule
    字号 & 命令 & 演示 \\
    \midrule
    初号 & \verb \zihao{0} & { \zihao{0} 初号 } \\
    一号 & \verb \zihao{1} & { \zihao{1} 一号 } \\
    小一 & \verb \zihao{-1} & { \zihao{-1} 小一 } \\
    二号 & \verb \zihao{2} & { \zihao{2} 二号 } \\
    小二 & \verb \zihao{-2} & { \zihao{-2} 小二 } \\
    三号 & \verb \zihao{3} & { \zihao{3} 三号 } \\
    小三 & \verb \zihao{-3} & { \zihao{-3} 小三 } \\
    四号 & \verb \zihao{4} & { \zihao{4} 四号 } \\
    小四 & \verb \zihao{-4} & { \zihao{-4} 小四 } \\
    五号 & \verb \zihao{5} & { \zihao{5} 五号 } \\
    小五 & \verb \zihao{-5} & { \zihao{-5} 小五 } \\
    \bottomrule
    \end{tabular}
    \end{table}

\section{字体}

    本模板西文主字体被设置为 Times New Roman,中文主字体被设置为中易宋体。在使用过程中如果需要调入其他字体,模板中提供了如下几种基本字体,如果不能满足需要,请自行搜索添加字体的方法。

    \subsection{中文字体}

    \begin{table}[H]
    \centering
    \topcaption{中文字体}
    \label{tab:font-zh}
    \begin{tabular}{lll}
    \toprule
    字体 & 命令 & 演示 \\
    \midrule
    宋体 & \verb \songti & { \songti 宋体 } \\
    黑体 & \verb \heiti & { \heiti 黑体 } \\
    仿宋 & \verb \fangsong & { \fangsong 仿宋 } \\
    隶书 & \verb \lishu & { \lishu 隶书 } \\
    楷书 & \verb \kaishu & { \kaishu 楷书 } \\
    \bottomrule
    \end{tabular}
    \end{table}

    \subsection{西文字体}

    \begin{table}[H]
    \centering
    \topcaption{西文字体}
    \label{tab:font-en}
    \begin{tabular}{lcr}
    \toprule
    字体 & 命令 & 演示 \\
    \midrule
    西文正体 &  & Times New Roman \\
    西文斜体 & \verb \emph{} & \emph{Times New Roman} \\
    \bottomrule
    \end{tabular}
    \end{table}

\section{三线表}

    \subsection{不指定行宽}

    \begin{table}[H]
    \centering
    \topcaption{对比表}
    \label{tab:tabsty-1}
    \begin{tabular}{lll}
    \toprule
    项目一 & 项目二 & 项目三 \\
    \midrule
     值1 & 值1 & 值1 \\
     值2 & 值2 & 值2 \\
     值3 & 值3 & 值3 \\
    \bottomrule
    \end{tabular}
    \end{table}

    \subsection{指定行宽}

    \begin{table}[H]
    \centering
    \topcaption{对比表}
    \label{tab:tabsty-2}
    \begin{tabular}{ p{2cm}<{\centering} p{2cm}<{\centering} p{2cm}<{\centering}  }
    \toprule
    项目一 & 项目二 & 项目三 \\
    \midrule
     值1 & 值1 & 值1 \\
     值2 & 值2 & 值2 \\
     值3 & 值3 & 值3 \\
    \bottomrule
    \end{tabular}
    \end{table}

    \subsection{调整表中字体大小}

    \begin{table}[H]
    \centering
    \topcaption{对比表}
    \label{tab:tabsty-3}
    \small
    \begin{tabular}{lcl}
    \toprule
    项目一 & 项目二 & 项目三 \\
    \midrule
     值1 & 值1 & 值1 \\
     值2 & 值2 & 值2 \\
     值3 & 值3 & 值3 \\
    \bottomrule
    \end{tabular}
    \end{table}

    \subsection{调整行高}

    \renewcommand\arraystretch{2}
    \begin{table}[H]
    \centering
    \topcaption{对比表}
    \label{tab:tabsty-4}
    \begin{tabular}{lcl}
    \toprule
    项目一 & 项目二 & 项目三 \\
    \midrule
     值1 & 值1 & 值1 \\
     值2 & 值2 & 值2 \\
     值3 & 值3 & 值3 \\
    \bottomrule
    \end{tabular}
    \end{table}

\section{插图}

    \subsection{单张图片}

     % 参数1为图片宽度,参数2为图片文件名,参数3位图片名称,参数4位该图片的引用标记
    \FIG{8cm}{1.jpg}{发动机}{fig:1}

    \subsection{并排图片}

     % 参数1为图片1的宽度,参数2为图片1的文件名,参数3位图片1的名称,参数4为图片1的引用标记
     % 参数5为图片2的宽度,参数6为图片5的文件名,参数7位图片2的名称,参数8位图片2的引用标记
    \TWOFIG{7cm}{1.jpg}{发动机1}{fig:2.1}{7cm}{2.jpg}{发动机2}{fig:2.2}

    \subsection{并排子图}

    % 参数1为图片1的名称,参数2为图片1的宽度,参数3位图片1的文件名,
    % 参数4为图片2的名称,参数2为图片5的宽度,参数6位图片2的文件名,
    % 参数7为图片总名称,参数8位图片的引用标记
    \SUBFIG{发动机1}{7cm}{1.jpg}{发动机2}{7cm}{2.jpg}{发动机}{fig:3}


\section{参考文献}

    例如,在 reference.tex 中的参考文献环境中有5条参考文献如下:

    \begin{verbatim}
    \bibitem{1} 薛华成.管理信息系统.北京:清华大学出版社,1993.
    \bibitem{2} 杨庆,栾茂田等.边坡渐进破坏可靠性分析及其应用.第八届土力学及
    岩土工程学术会议论文集.北京:万国学术出版社,1999.
    \bibitem{3} 徐滨士,欧忠文等.纳米表面工程.中国机械工程,2000,
    11(6):707-712.
    \bibitem{4} Kuehnlw M R, Peeken H, et al. The Toroidal Drive. Mechanical
    Engineering, 1981, 103 (2):32-39.
    \bibitem{5} 惠梦君,吴德海等.奥氏体—贝氏体球铁的发展.全国铸造学会奥氏体
    —贝氏体球铁专业学术会议,武汉,1986:201-205.
    \end{verbatim}

    如果在正文中想要引用该条文献,命令如下:

    \begin{verbatim}
    \upcite{文献索引号(填写在 reference.tex 中的文献)}
    \end{verbatim}

    我们想引用文献1、2、3、5,所以这样填写:

    \begin{verbatim}
    \upcite{1,2,3,5}
    \end{verbatim}

    示例:{ \color{red}{ 管理信息系统的组成很复杂\upcite{1,2,3,5}。 }}

\section{引用}

引用命令为 \verb \ref{图或表的引用标记} ,示例如下:

{\color{red}{ 如图\ref{fig:2.2}所示;如表\ref{tab:tabsty-1}所示。 }}


% !TeX root = ../report.tex

\chapter{实习报告的结构}

见《【教学(2017)8号】工程师阶段蓝领实习、技术实习答辩安排及要求通知》附件一。
\chapter{正文要求}

见《中欧航空工程师学院教学8号文件》附件一。


% !TeX root = ../report.tex

\chapter{规范表达注意事项}

见《【教学(2017)8号】工程师阶段蓝领实习、技术实习答辩安排及要求通知》附件一。

\chapter{结论}

见《【教学(2017)8号】工程师阶段蓝领实习、技术实习答辩安排及要求通知》附件一。
% !TeX root = ../report.tex

\bibliographystyle{plain}
\begin{thebibliography}{99}\addtolength{\itemsep}{-1.5ex}
\addcontentsline{toc}{chapter}{参考文献}
\zihao{-4}

\bibitem{1} 薛华成.管理信息系统.北京:清华大学出版社,1993.

\bibitem{2} 杨庆,栾茂田等.边坡渐进破坏可靠性分析及其应用.第八届土力学及岩土工程学术会议论文集.北京:万国学术出版社,1999.

\bibitem{3} 徐滨士,欧忠文等.纳米表面工程.中国机械工程,2000,11(6):707-712.

\bibitem{4} Kuehnlw M R, Peeken H, et al. The Toroidal Drive. Mechanical Engineering, 1981, 103 (2):32-39.

\bibitem{5} 惠梦君,吴德海等.奥氏体—贝氏体球铁的发展.全国铸造学会奥氏体—贝氏体球铁专业学术会议,武汉,1986:201-205.

\end{thebibliography} % 参考文献
% !TeX root = ../report.tex

\chapter*{RÉSUMÉ}
\addcontentsline{toc}{chapter}{RÉSUMÉ}

\lipsum[1-2] % RESUME
% !TeX root = ../report.tex

\chapter*{致谢}
\addcontentsline{toc}{chapter}{致谢}

用简短文字对在本研究工作中提出建议和给予帮助的人员,如老师和同学以及其他人,应在论文中做明确的说明并表示谢意。对导师的致谢要实事求是、诚恳、真挚,切忌滥用浮夸庸俗之词。
 % 致谢
% !TeX root = ../report.tex

\chapter*{附录A:蓝领实习周志}
\addcontentsline{toc}{chapter}{附录A:蓝领实习周志}


\chapter*{附录B:XXXXXX}
\addcontentsline{toc}{chapter}{附录B:XXXXX} % 附录
% !TeX root = ../report.tex

\newcommand\hbcol[3]{ \multicolumn{ #1 }{ | p{#2}<{\centering} | }{ #3 } } % 合并单元格

\newcommand{\tabincell}[2]{\begin{tabular}{@{}#1@{}}#2\end{tabular}} % 强制换行

\pagestyle{empty}

%------------------附件一---------------------------------------------
\clearpage
\newgeometry{top=3cm, bottom=2.5cm, left=2.7cm, right=2.7cm}
\begin{center}
\bfseries \zihao{2}
中国民航大学蓝领/技术实习单位鉴定表
\end{center}
% \addcontentsline{toc}{chapter}{中国民航大学蓝领/技术实习单位鉴定表}
\vspace*{-1mm}
\renewcommand\arraystretch{1.8}

\begin{table}[H]
\centering
\normalsize \kaishu
\begin{tabular}{ |c|c|c|c|c|c|c| }

\hline

\hbcol{2}{2cm}{实习课题} & \hbcol{5}{12cm}{} \\

\hline

\hbcol{2}{2cm}{实习单位} & \hbcol{2}{6cm}{} & \hbcol{2}{35mm}{实习起止时间} &  \\

\hline

\hbcol{2}{2cm}{实习类型} & \hbcol{2}{6cm}{ $\Box$ 蓝领实习 $\quad$ $\Box$ 技术实习 } & \hbcol{2}{35mm}{企业导师/负责人} &  \\

\hline

\hbcol{2}{2cm}{学生姓名} & \hspace*{3cm} & 专业 & \multicolumn{3}{|c|}{} \\

\hline

序号 & \hbcol{4}{10cm}{评价内容}  &  分值 & 得分 \\

\hline

1   &  \multirow{2}*{ \tabincell{c}{工作\\态度} }  & \multicolumn{3}{|p{9cm}|}{按期完成规定的任务,体现了本专业基本训练的内容} & 10 &  \\

\cline{1-1} \cline{3-7}

2   &                            & \multicolumn{3}{|p{9cm}|}{工作认真,遵守纪律,作风严谨务实} & 10 &  \\

\hline

3   &  \multirow{2}*{\tabincell{c}{工作\\投入}}    & \multicolumn{3}{|p{9cm}|}{严格遵守工作制度,保持足够的出勤率} & 10 &  \\

\cline{1-1} \cline{3-7}

4   &                            & \multicolumn{3}{|p{9cm}|}{精益求精,不忽视细节,积极改善工作方法} & 10 &  \\

\hline

5   &  \multirow{2}*{\tabincell{c}{工作\\绩效}}    & \multicolumn{3}{|p{9cm}|}{工作成果达到预期目的或计划要求} & 10 &  \\

\cline{1-1} \cline{3-7}

6   &                            & \multicolumn{3}{|p{9cm}|}{及时整理和总结工作成果,为以后的工作创造条件} & 10 &  \\

\hline

7   &  \multirow{4}*{\tabincell{c}{工作\\能力}}    & \multicolumn{3}{|p{9cm}|}{分析问题和解决问题的能力} & 10 &  \\

\cline{1-1} \cline{3-7}

8   &                            & \multicolumn{3}{|p{9cm}|}{充分的团队合作能力和协调沟通能力} & 10 &  \\

\cline{1-1} \cline{3-7}

9   &                            & \multicolumn{3}{|p{9cm}|}{充分的学习接受新知识和应用新知识的能力} & 10 &  \\

\cline{1-1} \cline{3-7}

10  &                            & \multicolumn{3}{|p{9cm}|}{具有创新意识,或有独特见解,有一定的应用价值} & 10 &  \\

\hline

\hbcol{5}{11cm}{总体评价}       & 100 &  \\

\hline

\multicolumn{7}{|p{15cm}|}{  评 语(请简要评价学生的实习过程,可包括:学生表现、工作完成情况、学生的优势与不足之处等。我们真诚期待您能对我们的学生或我们学院的教学及实习工作提出宝贵建议,以利于我们不断地进步): } \\

\multicolumn{7}{ |p{15cm}| }{  } \\
\multicolumn{7}{ |p{15cm}| }{  } \\
\multicolumn{7}{ |p{15cm}| }{  } \\

\multicolumn{7}{ |p{15cm}| }{  \hspace*{55mm} 企业导师/负责人签字: \hspace*{2cm} 年  \hspace*{4mm} 月 \hspace*{4mm} 日 } \\

\multicolumn{7}{ |p{15cm}| }{  } \\

\hline

\end{tabular}
\end{table}


















%------------------附件二---------------------------------------------

\newgeometry{top=3cm, bottom=2.5cm, left=2.7cm, right=2.7cm}
\clearpage
\begin{center}
\bfseries \zihao{2}
中国民航大学蓝领/技术实习教师评阅表
\end{center}
% \addcontentsline{toc}{chapter}{中国民航大学蓝领/技术实习教师评阅表}
\vspace*{-1mm}
\renewcommand\arraystretch{1.6}


\begin{table}[H]
\centering
\normalsize \kaishu
\begin{tabular}{ |c|c|c|c|c|c|c| }

\hline

\hbcol{2}{2cm}{实习课题} & \hbcol{5}{12cm}{} \\

\hline

\hbcol{2}{2cm}{实习单位} & \hbcol{2}{6cm}{} & \hbcol{2}{35mm}{实习起止时间} &  \\

\hline

\hbcol{2}{2cm}{实习类型} & \hbcol{5}{12cm}{ $\Box$ 蓝领实习 \hspace*{5cm} $\Box$ 技术实习 }  \\

\hline

\hbcol{2}{2cm}{学生姓名} & \hspace*{3cm} & 专业 & \multicolumn{3}{|c|}{} \\

\hline

\hbcol{2}{2cm}{评阅小组} & \hbcol{5}{12cm}{} \\

\hline

序号 & \hbcol{4}{10cm}{实习基本要求}  &  \multicolumn{2}{|c|}{是否满足} \\

\hline

1   & \hbcol{4}{10cm}{实习内容符合专业培养目标,圆满完成实习计划内容。}  &  \multicolumn{2}{|c|}{} \\

\hline

序号 & \hbcol{4}{10cm}{实习评价内容}  &  分值 & 得分 \\

\hline

1   &  \multirow{2}*{ \tabincell{c}{实习\\工作} }  & \multicolumn{3}{|p{9cm}|}{实习工作量、难度及周志记录情况} & 10 &  \\

\cline{1-1} \cline{3-7}

2   &                            & \multicolumn{3}{|p{9cm}|}{积极寻找实习内容与专业方向的联系,学以致用} & 5 &  \\

\hline

3   &  \multirow{2}*{\tabincell{c}{实习\\报告}}    & \multicolumn{3}{|p{9cm}|}{实习报告文稿规范性及三语摘要完成情况} & 10 &  \\

\cline{1-1} \cline{3-7}

4   &                            & \multicolumn{3}{|p{9cm}|}{问题综述充足、基本概念清晰、实际资料丰富具体} & 10 &  \\

\cline{1-1} \cline{3-7}

5   &                            & \multicolumn{3}{|p{9cm}|}{分析问题有逻辑,解决问题方案设计合理,具体可行} & 5 &  \\

\hline

6   &  \multirow{2}*{\tabincell{c}{实习\\海报}}    & \multicolumn{3}{|p{9cm}|}{海报制作规范,主要信息无缺漏} & 10 &  \\

\cline{1-1} \cline{3-7}

7   &                            & \multicolumn{3}{|p{9cm}|}{内容版式安排合理美观,实习工作内容突出} & 20 &  \\

\hline

8   &  \multirow{4}*{\tabincell{c}{实习\\答辩}}    & \multicolumn{3}{|p{9cm}|}{答辩PPT 制作规范,版面整洁,图文结合,重点突出} & 10 &  \\

\cline{1-1} \cline{3-7}

9   &                            & \multicolumn{3}{|p{9cm}|}{答辩过程表述清晰,具备一定的表达能力} & 10 &  \\

\cline{1-1} \cline{3-7}

10   &                            & \multicolumn{3}{|p{9cm}|}{对实习内容及报告理解透彻,回答问题简明扼要} & 10 &  \\

\hline


\hbcol{5}{11cm}{总体评价}       & 100 &  \\

\hline

\multicolumn{7}{|p{15cm}|}{  其它需要说明的内容: } \\

\multicolumn{7}{ |p{15cm}| }{ \hspace*{10cm} 组长签字: } \\
\multicolumn{7}{ |p{15cm}| }{ \hspace*{12cm} 年 \hspace*{4mm} 月 \hspace*{4mm} 日 } \\

\hline

\multicolumn{7}{ |c| }{ 实习成绩 } \\

\hline

\multicolumn{3}{|c|}{ \tabincell{c}{企业导师/负责人\\成绩(20\%)} } &   \multicolumn{2}{|c|}{  \tabincell{c}{  教师评阅\\成绩(80\%) } } & \multicolumn{2}{|c|}{总成绩} \\

\hline

\multicolumn{3}{|c|}{} &   \multicolumn{2}{|c|}{} & \multicolumn{2}{|c|}{} \\

\hline

\end{tabular}
\end{table}


















%------------------附件三---------------------------------------------

\newgeometry{top=3cm, bottom=2.5cm, left=2.7cm, right=2.7cm}
\newpage
\begin{center}
\bfseries \zihao{2}
中国民航大学蓝领/技术实习学生反馈表
\end{center}
% \chapter*{中国民航大学蓝领/技术实习学生反馈表}
% \addcontentsline{toc}{chapter}{中国民航大学蓝领/技术实习学生反馈表}
\vspace*{-1mm}
\renewcommand\arraystretch{1.6}


\begin{table}[H]
\centering
\normalsize \kaishu
\begin{tabular}{ |c|c|p{40mm}|p{12mm}|p{10mm}|p{10mm}|p{10mm}|p{10mm}| }

\hline

\hbcol{2}{2cm}{实习单位} &  & \hbcol{2}{22mm}{实习起止时间} & \hbcol{3}{30mm}{} \\

\hline

\hbcol{2}{2cm}{实习类型} & \multicolumn{6}{|c|}{ $\Box$ 蓝领实习 \hspace*{1cm} $\Box$ 技术实习 \hspace*{1cm} $\Box$ 专业实习 }  \\

\hline

\hbcol{2}{2cm}{学生姓名} &  & 专业 & \multicolumn{4}{|c|}{} \\

\hline

序号 & \multicolumn{2}{|c|}{评价内容}  & 很好 & 较好 & 好 & 一般 & 差 \\

\hline

1   &  \multirow{3}*{ \tabincell{c}{实习\\内容} }  & 实习内容目标明确,工作量及难度适宜 &  &  &  &  & \\

\cline{1-1} \cline{3-8}

2   &                                             & 实习时间长短合适,与实习内容适度匹配 &  &  &  &  & \\

\cline{1-1} \cline{3-8}

3   &                                             & 实习内容与本专业培养方向契合,利于专业能力成长。 &  &  &  &  & \\

\hline

4   &  \multirow{2}*{ \tabincell{c}{实习\\环境} }  & 能提供与实习内容相关的条件工具设施 &  &  &  &  & \\

\cline{1-1} \cline{3-8}

5   &                                             & 安全规定合理充足,安全工作到位 &  &  &  &  & \\


\hline


6   &  \multirow{2}*{ \tabincell{c}{实习\\指导} }  & 指导人员态度和蔼,积极认真 &  &  &  &  & \\

\cline{1-1} \cline{3-8}

7   &                                             & 所在部门氛围融洽,配合紧密 &  &  &  &  & \\


\hline

8   &  \tabincell{c}{实习\\效果}  & 能理解所在部门工作流程,对专业领悟提升明显 &  &  &  &  & \\

\hline

\multicolumn{3}{|c|}{总体评价}  &  &  &  &  & \\

\hline

\multicolumn{2}{|p{2cm}|}{其他需要说明的内容(如实习过程中遇到的问题,对学院或企业的建议等 }  & \multicolumn{6}{|c|}{} \\

\hline

\multicolumn{2}{|c|}{实习学生} &  & 日期 &  \multicolumn{4}{|c|}{}  \\

\hline

\end{tabular}
\end{table}
 % 附件


\end{document}
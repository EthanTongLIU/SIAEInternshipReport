% !TeX root = ../report.tex

\chapter{模板使用说明}

\section{模板简介}

本模板旨在为中欧学院同学提供一套标准规范的蓝领、技术、专业实习报告写作模板,减少同学们浪费在调整文献格式上的时间,故匆忙中制作了这套模板,模板按照《【教学(2017)8号】工程师阶段蓝领实习、技术实习答辩安排及要求通知》附件1所述要求制作,符合其所述规范,同学们可使用本模板生成实习报告,可以摆脱word排版的诸多弊端,譬如不稳定,格式调整繁琐等问题,同时提供了更加简便的特殊文字符号插入能力和多种定制环境支持。

\section{编译方法}

由于本模板中文处理采用 \verb xeCJK 宏包 ,所以本模板\textbf{\color{red}{必须使用 XeLaTeX 引擎编译}}。本模板是在 \verb texlive 下开发制作的,是目前最优的 \LaTeX 内核,故如果本模板在个人计算机上编译不通过,{\color{red}{\textbf{推荐安装} \verb texlive }}。本模板源文件采用 UTF8 编码,如果出现乱码,那是由于你采用了 GBK 编码的编译器,\textbf{\color{red}{希望你卸载 CTEX 套装,卸载 WinEdt 编辑器,使用我推荐的内核与编辑器}}。

{\bfseries 如何采用 XeLaTeX 引擎编译?}
\begin{itemize}
    \item 方法一:在将本机 \verb texlive 添加到环境变量的前提下,双击本模板源文件根目录下的  \verb compile&clean.bat 文件实现编译,如果编译不通过,说明你所输入的源码存在语法或其他错误,请认真检查后调试至通过。

    \item 方法二:使用本机自带 \LaTeX 编辑器,如 \verb SublimeText ,\verb Texmaker 等,自行切换至 XeLaTeX 编译引擎即可。
\end{itemize}


\section{源文件结构}

源文件根目录下包含三个文件夹及两个文档,对各个文件及文件夹的解释如下:

\begin{itemize}
    \item {\bfseries body}

    存放文章的各个章节,在编译时将本部分的文件自动调入主文档。每一章节分开编写,在本部分中填写。文件夹中还包括参考文献、附录、致谢、法语总结和附件。

    \item {\bfseries figure}

    图片路径,所有图片存入本文件夹中,在使用时自动调入。

    \item {\bfseries fonts}

    存放应用的字体。

    \item {\bfseries report.tex}

    主文件,在编译时应该编译此文件。

    \item {\bfseries SIAEInternshipReport.cls}

    定制的实习报告文档类,包含文档基本设置以及封面设计等操作。

\end{itemize}

本模板源文件拓扑结构如图所示:

\begin{Verbatim}[frame=single, framesep=5mm, samepage=false, baselinestretch=1.0]
root --|body --|chap1.tex
       |       |chap2.tex
       |       |chap3.tex
       |       |chap4.tex
       |       |chap5.tex
       |       |reference.tex (参考文献)
       |       |acknowlegement.tex (致谢)
       |       |appendix.tex (附录)
       |       |resume.tex (法文摘要)
       |       |attachement.tex (附件)
       |
       |figure (图片路径)
       |
       |fonts (字体)
       |
       |report.tex (主文件)
       |
       |SIAEInternshipReport.cls (定制文档类)
\end{Verbatim}


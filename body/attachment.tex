% !TeX root = ../report.tex

\newcommand\hbcol[3]{ \multicolumn{ #1 }{ | p{#2}<{\centering} | }{ #3 } } % 合并单元格

\newcommand{\tabincell}[2]{\begin{tabular}{@{}#1@{}}#2\end{tabular}} % 强制换行

\pagestyle{empty}

%------------------附件一---------------------------------------------
\clearpage
\newgeometry{top=3cm, bottom=2.5cm, left=2.7cm, right=2.7cm}
\begin{center}
\bfseries \zihao{2}
中国民航大学蓝领/技术实习单位鉴定表
\end{center}
% \addcontentsline{toc}{chapter}{中国民航大学蓝领/技术实习单位鉴定表}
\vspace*{-1mm}
\renewcommand\arraystretch{1.8}

\begin{table}[H]
\centering
\normalsize \kaishu
\begin{tabular}{ |c|c|c|c|c|c|c| }

\hline

\hbcol{2}{2cm}{实习课题} & \hbcol{5}{12cm}{} \\

\hline

\hbcol{2}{2cm}{实习单位} & \hbcol{2}{6cm}{} & \hbcol{2}{35mm}{实习起止时间} &  \\

\hline

\hbcol{2}{2cm}{实习类型} & \hbcol{2}{6cm}{ $\Box$ 蓝领实习 $\quad$ $\Box$ 技术实习 } & \hbcol{2}{35mm}{企业导师/负责人} &  \\

\hline

\hbcol{2}{2cm}{学生姓名} & \hspace*{3cm} & 专业 & \multicolumn{3}{|c|}{} \\

\hline

序号 & \hbcol{4}{10cm}{评价内容}  &  分值 & 得分 \\

\hline

1   &  \multirow{2}*{ \tabincell{c}{工作\\态度} }  & \multicolumn{3}{|p{9cm}|}{按期完成规定的任务,体现了本专业基本训练的内容} & 10 &  \\

\cline{1-1} \cline{3-7}

2   &                            & \multicolumn{3}{|p{9cm}|}{工作认真,遵守纪律,作风严谨务实} & 10 &  \\

\hline

3   &  \multirow{2}*{\tabincell{c}{工作\\投入}}    & \multicolumn{3}{|p{9cm}|}{严格遵守工作制度,保持足够的出勤率} & 10 &  \\

\cline{1-1} \cline{3-7}

4   &                            & \multicolumn{3}{|p{9cm}|}{精益求精,不忽视细节,积极改善工作方法} & 10 &  \\

\hline

5   &  \multirow{2}*{\tabincell{c}{工作\\绩效}}    & \multicolumn{3}{|p{9cm}|}{工作成果达到预期目的或计划要求} & 10 &  \\

\cline{1-1} \cline{3-7}

6   &                            & \multicolumn{3}{|p{9cm}|}{及时整理和总结工作成果,为以后的工作创造条件} & 10 &  \\

\hline

7   &  \multirow{4}*{\tabincell{c}{工作\\能力}}    & \multicolumn{3}{|p{9cm}|}{分析问题和解决问题的能力} & 10 &  \\

\cline{1-1} \cline{3-7}

8   &                            & \multicolumn{3}{|p{9cm}|}{充分的团队合作能力和协调沟通能力} & 10 &  \\

\cline{1-1} \cline{3-7}

9   &                            & \multicolumn{3}{|p{9cm}|}{充分的学习接受新知识和应用新知识的能力} & 10 &  \\

\cline{1-1} \cline{3-7}

10  &                            & \multicolumn{3}{|p{9cm}|}{具有创新意识,或有独特见解,有一定的应用价值} & 10 &  \\

\hline

\hbcol{5}{11cm}{总体评价}       & 100 &  \\

\hline

\multicolumn{7}{|p{15cm}|}{  评 语(请简要评价学生的实习过程,可包括:学生表现、工作完成情况、学生的优势与不足之处等。我们真诚期待您能对我们的学生或我们学院的教学及实习工作提出宝贵建议,以利于我们不断地进步): } \\

\multicolumn{7}{ |p{15cm}| }{  } \\
\multicolumn{7}{ |p{15cm}| }{  } \\
\multicolumn{7}{ |p{15cm}| }{  } \\

\multicolumn{7}{ |p{15cm}| }{  \hspace*{55mm} 企业导师/负责人签字: \hspace*{2cm} 年  \hspace*{4mm} 月 \hspace*{4mm} 日 } \\

\multicolumn{7}{ |p{15cm}| }{  } \\

\hline

\end{tabular}
\end{table}


















%------------------附件二---------------------------------------------

\newgeometry{top=3cm, bottom=2.5cm, left=2.7cm, right=2.7cm}
\clearpage
\begin{center}
\bfseries \zihao{2}
中国民航大学蓝领/技术实习教师评阅表
\end{center}
% \addcontentsline{toc}{chapter}{中国民航大学蓝领/技术实习教师评阅表}
\vspace*{-1mm}
\renewcommand\arraystretch{1.6}


\begin{table}[H]
\centering
\normalsize \kaishu
\begin{tabular}{ |c|c|c|c|c|c|c| }

\hline

\hbcol{2}{2cm}{实习课题} & \hbcol{5}{12cm}{} \\

\hline

\hbcol{2}{2cm}{实习单位} & \hbcol{2}{6cm}{} & \hbcol{2}{35mm}{实习起止时间} &  \\

\hline

\hbcol{2}{2cm}{实习类型} & \hbcol{5}{12cm}{ $\Box$ 蓝领实习 \hspace*{5cm} $\Box$ 技术实习 }  \\

\hline

\hbcol{2}{2cm}{学生姓名} & \hspace*{3cm} & 专业 & \multicolumn{3}{|c|}{} \\

\hline

\hbcol{2}{2cm}{评阅小组} & \hbcol{5}{12cm}{} \\

\hline

序号 & \hbcol{4}{10cm}{实习基本要求}  &  \multicolumn{2}{|c|}{是否满足} \\

\hline

1   & \hbcol{4}{10cm}{实习内容符合专业培养目标,圆满完成实习计划内容。}  &  \multicolumn{2}{|c|}{} \\

\hline

序号 & \hbcol{4}{10cm}{实习评价内容}  &  分值 & 得分 \\

\hline

1   &  \multirow{2}*{ \tabincell{c}{实习\\工作} }  & \multicolumn{3}{|p{9cm}|}{实习工作量、难度及周志记录情况} & 10 &  \\

\cline{1-1} \cline{3-7}

2   &                            & \multicolumn{3}{|p{9cm}|}{积极寻找实习内容与专业方向的联系,学以致用} & 5 &  \\

\hline

3   &  \multirow{2}*{\tabincell{c}{实习\\报告}}    & \multicolumn{3}{|p{9cm}|}{实习报告文稿规范性及三语摘要完成情况} & 10 &  \\

\cline{1-1} \cline{3-7}

4   &                            & \multicolumn{3}{|p{9cm}|}{问题综述充足、基本概念清晰、实际资料丰富具体} & 10 &  \\

\cline{1-1} \cline{3-7}

5   &                            & \multicolumn{3}{|p{9cm}|}{分析问题有逻辑,解决问题方案设计合理,具体可行} & 5 &  \\

\hline

6   &  \multirow{2}*{\tabincell{c}{实习\\海报}}    & \multicolumn{3}{|p{9cm}|}{海报制作规范,主要信息无缺漏} & 10 &  \\

\cline{1-1} \cline{3-7}

7   &                            & \multicolumn{3}{|p{9cm}|}{内容版式安排合理美观,实习工作内容突出} & 20 &  \\

\hline

8   &  \multirow{4}*{\tabincell{c}{实习\\答辩}}    & \multicolumn{3}{|p{9cm}|}{答辩PPT 制作规范,版面整洁,图文结合,重点突出} & 10 &  \\

\cline{1-1} \cline{3-7}

9   &                            & \multicolumn{3}{|p{9cm}|}{答辩过程表述清晰,具备一定的表达能力} & 10 &  \\

\cline{1-1} \cline{3-7}

10   &                            & \multicolumn{3}{|p{9cm}|}{对实习内容及报告理解透彻,回答问题简明扼要} & 10 &  \\

\hline


\hbcol{5}{11cm}{总体评价}       & 100 &  \\

\hline

\multicolumn{7}{|p{15cm}|}{  其它需要说明的内容: } \\

\multicolumn{7}{ |p{15cm}| }{ \hspace*{10cm} 组长签字: } \\
\multicolumn{7}{ |p{15cm}| }{ \hspace*{12cm} 年 \hspace*{4mm} 月 \hspace*{4mm} 日 } \\

\hline

\multicolumn{7}{ |c| }{ 实习成绩 } \\

\hline

\multicolumn{3}{|c|}{ \tabincell{c}{企业导师/负责人\\成绩(20\%)} } &   \multicolumn{2}{|c|}{  \tabincell{c}{  教师评阅\\成绩(80\%) } } & \multicolumn{2}{|c|}{总成绩} \\

\hline

\multicolumn{3}{|c|}{} &   \multicolumn{2}{|c|}{} & \multicolumn{2}{|c|}{} \\

\hline

\end{tabular}
\end{table}


















%------------------附件三---------------------------------------------

\newgeometry{top=3cm, bottom=2.5cm, left=2.7cm, right=2.7cm}
\newpage
\begin{center}
\bfseries \zihao{2}
中国民航大学蓝领/技术实习学生反馈表
\end{center}
% \chapter*{中国民航大学蓝领/技术实习学生反馈表}
% \addcontentsline{toc}{chapter}{中国民航大学蓝领/技术实习学生反馈表}
\vspace*{-1mm}
\renewcommand\arraystretch{1.6}


\begin{table}[H]
\centering
\normalsize \kaishu
\begin{tabular}{ |c|c|p{40mm}|p{12mm}|p{10mm}|p{10mm}|p{10mm}|p{10mm}| }

\hline

\hbcol{2}{2cm}{实习单位} &  & \hbcol{2}{22mm}{实习起止时间} & \hbcol{3}{30mm}{} \\

\hline

\hbcol{2}{2cm}{实习类型} & \multicolumn{6}{|c|}{ $\Box$ 蓝领实习 \hspace*{1cm} $\Box$ 技术实习 \hspace*{1cm} $\Box$ 专业实习 }  \\

\hline

\hbcol{2}{2cm}{学生姓名} &  & 专业 & \multicolumn{4}{|c|}{} \\

\hline

序号 & \multicolumn{2}{|c|}{评价内容}  & 很好 & 较好 & 好 & 一般 & 差 \\

\hline

1   &  \multirow{3}*{ \tabincell{c}{实习\\内容} }  & 实习内容目标明确,工作量及难度适宜 &  &  &  &  & \\

\cline{1-1} \cline{3-8}

2   &                                             & 实习时间长短合适,与实习内容适度匹配 &  &  &  &  & \\

\cline{1-1} \cline{3-8}

3   &                                             & 实习内容与本专业培养方向契合,利于专业能力成长。 &  &  &  &  & \\

\hline

4   &  \multirow{2}*{ \tabincell{c}{实习\\环境} }  & 能提供与实习内容相关的条件工具设施 &  &  &  &  & \\

\cline{1-1} \cline{3-8}

5   &                                             & 安全规定合理充足,安全工作到位 &  &  &  &  & \\


\hline


6   &  \multirow{2}*{ \tabincell{c}{实习\\指导} }  & 指导人员态度和蔼,积极认真 &  &  &  &  & \\

\cline{1-1} \cline{3-8}

7   &                                             & 所在部门氛围融洽,配合紧密 &  &  &  &  & \\


\hline

8   &  \tabincell{c}{实习\\效果}  & 能理解所在部门工作流程,对专业领悟提升明显 &  &  &  &  & \\

\hline

\multicolumn{3}{|c|}{总体评价}  &  &  &  &  & \\

\hline

\multicolumn{2}{|p{2cm}|}{其他需要说明的内容(如实习过程中遇到的问题,对学院或企业的建议等 }  & \multicolumn{6}{|c|}{} \\

\hline

\multicolumn{2}{|c|}{实习学生} &  & 日期 &  \multicolumn{4}{|c|}{}  \\

\hline

\end{tabular}
\end{table}
